\begin{figure}[H]
  \begin{inference}
  \jmpInfer{BaseType-T}
      {width \in \mathbb{N} \andalso lanes \in \mathbb{N}}
      {{\begin{array}{l}
        \Delta \vdash \kwd{int} width \kwd{x} lanes : \texttt{BaseType} \\
        \Delta \vdash \kwd{uint} width \kwd{x} lanes : \texttt{BaseType} \\
        \Delta \vdash \kwd{float} width \kwd{x} lanes : \texttt{BaseType} \\
        \Delta \vdash \kwd{bool} lanes : \texttt{BaseType} \\
      \end{array}}}
  \jmpInfer{Shape-T}
    {\forall {i \in [1, n]}\colon \, d_i \in \mathbb{N}}
    {\Delta \vdash \kwd{(} d_1 \kwd{,} d_2 \kwd{,} \ldots \kwd{,} d_n \kwd{)} : \texttt{Shape} }
  \jmpInfer{Relation-T}
    {\Delta, T_1 : \texttt{Type}, \ldots, T_n : \texttt{Type} \vdash (Rel(T_1, T_2, \ldots, T_n) \in \{ \top, \bot \}) }
    {\Delta \vdash Rel : \texttt{Relation}}
  \jmpInfer{Tensor-T}
    {\Delta \vdash bt : \texttt{BaseType} \andalso \Delta \vdash s : \texttt{Shape}}
    {\Delta \vdash \kwd{Tensor[}s\kwd{,} bt\kwd{]} : \texttt{Type} }
  \jmpInfer{Function-T}
    {\forall {i \in [1, r]}\colon \, \Delta \vdash R_i : \texttt{Relation}
    \\ \forall {i \in [1, m]}\colon \, \Delta \vdash T_i : \texttt{Type}
    \\ \forall {i \in [1, r]}\colon \, \Delta, T_{m+1} : \texttt{Type}, \ldots, T_{m+n} : \texttt{Type}
    \\ \hspace{1.5em} \vdash R_i(T_1, \ldots, T_m, O)[T_{m + 1} / V_1, \ldots, T_{m+n} / V_n] }
    {\Delta \vdash \kwd{fn$\langl$} V_1 \kwd{,} \ldots\kwd{,} V_n \kwd{$\rangl$(} T_1 \kwd{,} \ldots \kwd{,} T_m
    \kwd{) $\rightarrow$ } O \\ \kwd{ where } R_1, \ldots, R_r : \texttt{Type} }
    \end{inference}
  \end{figure}
  \begin{figure}[H]
    \ContinuedFloat
    \begin{inference}
  \jmpInfer{Product-T}
    {\forall i \in [1,n] \colon \, \Delta \vdash T_i : \texttt{Type}}
    {\Delta \vdash \kwd{(} T_1 \kwd{,} T_2 \kwd{,} \ldots \kwd{,} T_n\kwd{)} : \texttt{Type} }
  \jmpInfer{Ref-T}
    {\Delta \vdash T : \texttt{Type}}
    {\Delta \vdash \texttt{RefType} \kwd{[} T \kwd{]} : \texttt{Type}}
  \jmpInfer{ADT-T}
    {\forall{i \in [1, n]} \, \Delta, T_{m+1} : \texttt{Type}, \ldots, T_{m+k} : \texttt{Type}, S : \texttt{ADT}
      \\ \hspace{1.5em} \vdash T_i[T_{m + 1} / V_1, \ldots, T_{m + k} / V_k] : \texttt{Type}
      \\ S \mapsto \kwd{data} \langle V_1, \ldots, V_k \rangle \kwd{\{}C_1(T_1, \ldots, T_n) | \ldots | C_m(T_1, \ldots, T_n)\kwd{\}} \in \Delta}
    {\Delta \vdash S : \texttt{ADT}}
  \jmpInfer{TypeCall-T}
      {\Delta \vdash S : \texttt{ADT} \andalso S \mapsto \kwd{data} \langle V_1, \ldots, V_k \rangle \kwd{\{} \ldots \kwd{\}} \in \Delta
      \\ \Delta \vdash T_1 : \texttt{Type}, \ldots, T_k : \texttt{Type}}
  {\Delta \vdash S \kwd{[} T_1, \ldots, T_k \kwd{]} : \texttt{Type}}
\end{inference}
  \caption{Rules for constructing types, indicating kinds. Reference types are only generated internally by reverse-mode automatic differentiation and cannot be given in front-end user code. Relations cannot be defined in front-end user code either, and instead must be implemented and registered with operations. For simplicity, the rule for ADT definitions assumes that each ADT constructor takes the same $n$ arguments (whereas each constructor may take any number of arguments, so long as they are all of kind \texttt{Type}) and the rule for function types assumes that each relation $R$ takes the same $n$ arguments, whereas a relation may take any subset of the set of all type arguments, argument types, and the output type as arguments (so long as they are all of kind \texttt{Type}).
Note that the kind \texttt{ADT} corresponds to an ADT \textit{name} (implemented as a global type variable);
any ADT instance must instantiate all type parameters in the ADT definition, hence the use of a type call for
giving a type to ADT instances (that is, an ADT definition is treated as a type-level function).}
  \label{fig:kind-rules}
  \end{figure}

  \begin{figure}[H]
  \begin{inference}
  \jmpInfer{Type-Constant}
      {\Delta \vdash s : \texttt{Shape} \andalso \Delta \vdash bt : \texttt{BaseType} \andalso
        v \in \mathbb{R} \cup \{\texttt{True}, \texttt{False}\}}
      {\Delta; \Gamma \vdash \kwd{const}(v, s, bt) : \kwd{Tensor[}s, bt\kwd{]}}
  \jmpInfer{Type-Product}
    {\forall i \in [1, n]\colon \, \Delta; \Gamma \vdash p_i : T_i}
    {\Delta; \Gamma \vdash \kwd{(}p_1 \kwd{,} \ldots\kwd{,} p_n\kwd{)} : \kwd{(} T_1
    \kwd{,}\ldots \kwd{,} T_n \kwd{)} }
  \jmpInfer{Type-Projection}
    {\Delta; \Gamma \vdash p : \kwd{(} T_1
    \kwd{,}\ldots \kwd{,} T_n \kwd{)} \\ i \in \lbrack 0, n)}
    {\Delta; \Gamma \vdash p \kwd{.} i : T_{i+1}}
  \jmpInfer{Type-Let}
    {\Delta; \Gamma \vdash v : T \andalso \Delta; \Gamma, id : T \vdash e : T^\prime}
    {\Delta; \Gamma \vdash \kwd{let \%} id\ \kwd{=}\ v \kwd{;}\ e : T^\prime}
  \jmpInfer{Type-Ref}
    {\Delta; \Gamma \vdash n : T}
    {\Delta; \Gamma \vdash \texttt{Ref}\ n : \texttt{RefType} \kwd{[} T \kwd{]} }
  \jmpInfer{Type-Read-Ref}
    {\Delta; \Gamma \vdash r : \texttt{RefType} \kwd{[} T \kwd{]}}
    {\Delta; \Gamma \vdash\ !r : T }
  \jmpInfer{Type-Write-Ref}
    {\Delta; \Gamma \vdash r : \texttt{RefType} \kwd{[} T \kwd{]} \andalso \Delta; \Gamma \vdash v : T}
    {\Delta; \Gamma \vdash r := v : () }
  \jmpInfer{Type-If}
    {\Delta; \Gamma \vdash c : \kwd{Tensor[}(), \kwd{bool}(1)\kwd{]} \\ \Delta; \Gamma \vdash b_1 : T \andalso \Delta; \Gamma \vdash b_2 : T}
    {\Delta; \Gamma \vdash \kwd{if}\ c\ \kwd{then \{}\ b_1\ \kwd{\} else \{}\ b_2\ \kwd{\}} : T}
  \jmpInfer{Type-Gradient}
    {\Delta; \Gamma \vdash f : \kwd{fn$\langl\rangl$(}T_1, \ldots, T_n\kwd{) $\rightarrow$ } O}
    {\Delta; \Gamma \vdash \texttt{Grad}\ f : \kwd{fn$\langl\rangl$(}T_1, \ldots, T_n\kwd{) $\rightarrow$ } (O, (T_1, \ldots, T_n))}
  \jmpInfer{Type-ADT-Constructor}
    {{\Delta \vdash S : \texttt{ADT}
      \\ S \mapsto \kwd{data} \langl V_1, \ldots, V_k \rangl \kwd{\{} C_1(T_1, \ldots, T_n) | \ldots | C_m(T_1, \ldots, T_n) \kwd{\}} \in \Delta}}
    {\forall i \in [1, m] \, \Delta; \Gamma \vdash C_i : \kwd{fn} \langl V_1, \ldots, V_k \rangl (T_1, \ldots, T_n)
      \rightarrow S \kwd{[} V_1, \ldots, V_k \kwd{]} }
  \end{inference}
\end{figure}
\begin{figure}[H]
  \ContinuedFloat
  \begin{inference}
  \jmpInfer{Type-Function-Definition}
    {\Delta \vdash R_1 : \texttt{Relation}, \ldots, R_r : \texttt{Relation}
      \\ \Delta \vdash T_1 : \texttt{Type}, \ldots, T_m : \texttt{Type}, O :
      \texttt{Type}
      \\ {\begin{array}{l}
            \hspace{-0.5em} \Delta, T_{m+1} : \texttt{Type}, \ldots, T_{m+n} : \texttt{Type}, \\
            \hspace{1.0em} (\forall {i \in [1, r]} \, R_i(T_1, \ldots, T_m, O) [T_{m + 1} / V_1, \ldots, T_{m + n} / V_n]); \\
            \hspace{0.5em} \Gamma, (\forall {i \in [1, m]}, a_i : T_i [T_{m + 1} / V_1, \ldots, T_{m + n} / V_n]), \\
            \hspace{2.0em} f : \kwd{fn$\langl$} V_1 \kwd{,} \ldots\kwd{,} V_n \kwd{$\rangl$(} T_1 \kwd{,} \ldots \kwd{,} T_m
            \kwd{)} \rightarrow O \\
            \hspace{3.0em} \kwd{ where } R_1, \ldots, R_r \\
            \hspace{1.0em} \vdash body : O[T_{m + 1} / V_1, \ldots, T_{m + n} / V_n]
          \end{array}}
    }
    {\Delta; \Gamma \vdash \kwd{def @} f \kwd{$\langl$} V_1\kwd{,} \ldots\kwd{,} V_n
    \kwd{$\rangl$} \kwd{(}a_1\kwd{:} T_1\kwd{,} \ldots a_m\kwd{:} T_m\kwd{) $\rightarrow$ } O
    \\ \kwd{ where } R_1, \ldots, R_r \kwd{ \{ } body \kwd{ \}} : \\ \kwd{fn$\langl$} T_{m+1} \kwd{,} \ldots\kwd{,} T_n \kwd{$\rangl$(} T_1 \kwd{,} \ldots \kwd{,} T_m
    \kwd{) $\rightarrow$ } O \\ \kwd{ where } R_1, \ldots, R_r }
  \jmpInfer{Type-Call}
    {\\ \Delta; \Gamma \vdash f : \kwd{fn$\langl$} V_1 \kwd{,} \ldots\kwd{,} V_n \kwd{$\rangl$(} T_1 \kwd{,} \ldots \kwd{,} T_m
    \kwd{) $\rightarrow$ } O \\ \kwd{ where } R_1, \ldots, R_r
    \\ \Delta \vdash T_{m+1} : \texttt{Type}, \ldots, T_n : \texttt{Type}
    \\ \forall{i \in [1, m]} \, \Delta; \Gamma \vdash a_i : T_i[\forall{j \in [1, n]} \, T_{m + j} / V_j]
    \\ \forall{i \in [1, r]} \, \Delta; \Gamma \vdash R_i(T_1, \ldots, T_m, O)[\forall{j \in [1, n]} \, T_{m + j} / V_j]}
    {\Delta; \Gamma \vdash f\kwd{$\langl$} T_{m+1}, \ldots, T_{m+n} \kwd{$\rangl$}(a_1, \ldots, a_m) : O [\forall{i \in [1, n]} \, T_{m + i} / V_i]}
  \jmpInfer{Type-ADT-Match}
    {\\ \Delta \vdash S : \texttt{ADT}, T_{m+1} : \texttt{Type}, \ldots,
    T_{m+k} : \texttt{Type}
      \\ S \mapsto \kwd{data} \langl V_1, \ldots, V_k \rangl
          \kwd{\{} C_1(T_1, \ldots, T_n) | \ldots | C_m(\ldots) \kwd{\}} \in \Delta
      \\ \forall i \in [1, m] \, \Delta; \Gamma, (\forall j \in [1, n] \, a_j : T_{j+k} [T_1 / V_1, \ldots, T_k / V_k]) \vdash e_i : T^\prime
      \\ \Delta; \Gamma \vdash s : S\kwd{[}T_1, \ldots, T_k\kwd{]} }
    {\Delta; \Gamma \vdash
    {\begin{array}{l}
      \hspace{-0.5em} \kwd{match(}s\kwd{) \{} \\
      \hspace{0.5em} \kwd{|} \, C_1 \langl T_{m+1}, \ldots, T_{m+k} \rangl \kwd{(}a_1\kwd{,} \ldots\kwd{,} a_n\kwd{) => } e_1 \\
      \hspace{4.5em} \vdots \\
      \hspace{0.5em} \kwd{|} \, C_m \langl T_{m+1}, \ldots, T_{m+k} \rangl
      \kwd{(}a_1\kwd{,} \ldots\kwd{,} a_n\kwd{) => } e_m \\
      \hspace{-0.5em} \kwd{\}}
    \end{array}}
    : T^\prime}
  \end{inference}
      \caption{Rules for deriving types of expressions and definitions. The unit type, $()$, is syntactic sugar
              for a product type with zero members. For simplicity, the arguments to ADT constructors have been omitted in the
              descriptions of the ADTs and each constructor is assumed to take the same $n$ argument types (similarly, the names
              of the match parameters in the rule for match expressions are assumed to be the same in each branch. Relay{} also supports
              more sophisticated pattern-matching in match expressions but these rules omit this for simplicity).
              Type arguments and relations are omitted in the rule for gradient, as the present implementation of AD does not support them.
              Note that ADT constructors are given ordinary function types and can thus be called according to the same rules
              as any other function.}
      \label{fig:inference-rules}
 \end{figure}
