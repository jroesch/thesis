\begin{fullwidth}
\begin{center}
\raggedright\setlength{\parindent}{1em}
In the past decade \emph{deep learning} has revolutionized many areas of
    computer science.
Deep learning computations mostly consist
    of expensive linear algebra kernels defined over a mixture
    of large sparse and dense tensors.
From the early days of deep learning framework development, researchers realized the
    potential for applying compiler optimizations to accelerate neural networks.
As deep learning continues to grow in popularity the diversity
    of models also continues to grow.
Due to the early success of deep learning in computer
    vision, early deep learning systems were were focused on static,
    feed-forward networks processing fixed sized images.
First-generation deep learning compilers have also been similarly overfit
    for static model compilation, with strong assumptions of static control-flow,
    static tensor dimensions and no complex data structures.
A focus on static models has created challenges for deep learning practitioners,
    as dynamic models introduce input-dependent graph topology,
    violating key invariants of existing systems and invalidating optimizations
    designed for purely static data flow graphs.
This lack of support has manifested as series of ad-hoc extensions to
  both frameworks, deep learning runtimes and compilers.
Choosing to ignore dynamic behaviors has allowed deep learning
    compilers to make significant strides in optimizing common
    deep learning workloads, but existing techniques miss
    increasing generality without sacrificing performance.

This dissertation in particular focuses on an under served, yet important problem:
  the representation,
  optimization,
  differentiation,
  and execution of \emph{dynamic neural networks}.
In this thesis I propose generalizing overspecialized
  compilation techniques applied to static dataflow graphs,
  the predominant programming model of deep learning,
  to fully dynamic neural networks.
These generalizations are powered by a simple insight:
  dynamic neural networks are just programs which manipulate tensors.
The challenge is constructing a representation that captures this generality
  in a principled manner, while not sacrificing state-of-the-art performance or the programming model.
In particular, the contributions include:
\begin{itemize}
    \item An intermediate representation which can represent dynamic behaviors.
    \item A new automatic differentiation technique for dynamic neural networks.
    \item A set of general optimizations which work on all programs, as well
          as specialized dynamic optimizations.
    \item An efficient runtime for dynamic neural networks.
\end{itemize}

The efforts of my thesis now exists in Apache TVM, a deep learning compiler framework.
Apache TVM is deployed in production at multiple leading companies including
  Amazon, Facebook, and Microsoft and is a critical piece of the technology stack
  at OctoML: a company I co-founded around the TVM project.
One notable impact is its use in Amazon Alexa, Amazon's AI assistant
  which executes on a variety of devices such as ``smart speakers'' which include
  digital assistants.
Amazon engineers used Relay to optimize Alexa’s wake word model,
  executed each time a user interacts with Alexa.

\end{center}
\end{fullwidth}
