\chapter{Executing Relay}
\label{ch:execute}

The final piece of the puzzle is taking the sufficiently
    lowered Relay program and executing it.
In theory a naive interpretation strategy could be applied
    to the entire Relay program.

In the below section we describe each of the execution
    mechanisms of Relay and how a user program flows
    from user specified programs to the final result.

\section{Compiler Framework}

The process for compiling Relay proceeds in three stages.
First, the frontend converts input formats into the Relay IR.
Next, the Relay compiler typechecks and optimizes the program
    to produce the final program.
After performing optimizations,
    the Relay backend transforms
    the Relay program into a form that can be executed on
    the intended hardware, based on the specified execution mechanism.
The backend additionally lowers Relay operators into a TVM expression,
    computes a schedule for the final TVM expression, and lowers it into
    native code.

\subsection{Frontend}

There are several ways to write an Relay program.
A user can build an in-memory representation of
    a program in C++ or Python,
    parse one written in the Relay text format,
    load one from the on-disk serialization format,
    or import one from popular frameworks and interchange formats
    (e.g., TensorFlow, MxNet, Keras, DarkNet, and ONNX).
Many frameworks and interchange formats use static computation graph-based representations,
    which can easily be translated into Relay.
A greater challenge is translating frameworks
    with a richer computation model such as TensorFlow (TF).
TF supports control flow and includes \verb|TensorArray|, a write-once
    tensor container.
We can extract the loop structure out of the TF graph, converting
    it to an Relay loop, and transform the \verb|TensorArray| into an Relay list.
Once new deep learning languages and IRs under development
    are stable it is likely they can be translated into Relay (see
    Section~\ref{sec:pl_techniques_in_dl}).
PyTorch provides an expressive programming model, and is a good fit
    for Relay, which has integration into PyTorch's JIT infrastructure,
    enabling users to transparently use Relay for improved performance.

\subsection{Frontend}

There are several ways to write an Relay program.
A user can build an in-memory representation of
    a program in C++ or Python,
    parse one written in the Relay text format,
    load one from the on-disk serialization format,
    or import one from popular frameworks and interchange formats
    (e.g., TensorFlow, MxNet, Keras, DarkNet, and ONNX).
Many frameworks and interchange formats use static computation graph-based representations,
    which can easily be translated into Relay.
A greater challenge is translating frameworks
    with a richer computation model such as TensorFlow (TF).
TF supports control flow and includes \verb|TensorArray|, a write-once
    tensor container.
We can extract the loop structure out of the TF graph, converting
    it to an Relay loop, and transform the \verb|TensorArray| into an Relay list.
Once new deep learning languages and IRs under development
    are stable it is likely they can be translated into Relay (see
    Section~\ref{sec:pl_techniques_in_dl}).
PyTorch provides an expressive programming model, and is a good fit
    for Relay, which has integration into PyTorch's JIT infrastructure,
    enabling users to transparently use Relay for improved performance.


    \subsection{Frontend}

    There are several ways to write an Relay program.
    A user can build an in-memory representation of
      a program in C++ or Python;
      parse one written in the Relay text format;
      or load one from the on-disk serialization format,
      similar in design to LLVM's bitcode.
    Models from popular frameworks, including
      TensorFlow, PyTorch, MxNet, Keras, and DarkNet, as well as interchange
      formats, such as ONNX, may be imported directly into Relay.
    \subsection{Compiler}
    Once an Relay abstract syntax tree (AST) is produced,
      the program is optimized by applying a series of Relay-to-Relay
      passes.
    Between each pass, Relay performs type inference and checking,
      rejecting malformed programs as well as populating shape and type
      information that passes can utilize.
    Relay optimizations consist of both traditional compiler
      optimizations as well as domain-specific optimizations.
    Traditional compiler optimizations include constant folding,
      common subexpression elimination,
      and dead code elimination.
    DL-specific optimizations include
      operator fusion,
      quantization,
      layout transformation,
      and accelerator-specific optimizations.

    Relay produces machine-specific code
      by decomposing the problem of code generation into multiple distinct phases.
    % See Figure~\ref{fig:pipeline} for a visual overview of each stage.
    Since Relay is a high-level IR, it depends on a low-level code generator,
      such as \tvm or Halide,
      to produce dense linear algebra kernels~\citep{tvm_osdi18, halide}.
    We use \tvm in our experiments.
    Low-level kernel compilers focus on generating highly efficient operators.
    The generated kernels have a fixed calling convention and do not
      handle allocation. Instead, they expect inputs and outputs to be preallocated.
    From an optimized AST,
      the compiler extracts a set of Relay operators,
      translates them to TVM expressions,
      and then compiles to available hardware targets.
    The resulting output is an
      object file that contains the compiled operators
      and an Relay program that invokes these primitives.
    In our prototype implementation,
      we are able to target CPU, GPU,
      iOS and Android mobile devices,
      custom accelerators, and FPGAs.

\subsection{Compiler}
Once an Relay abstract syntax tree (AST) is produced,
    the program is optimized by applying a series of Relay-to-Relay
    passes.
Between each pass, Relay performs type inference and checking,
    rejecting malformed programs as well as populating shape and type
    information that passes can utilize.
The Relay compiler supports traditional optimizations
    (e.g., constant folding, common subexpression elimination, and dead code elimination)
    and domain-specific optimizations
    (see Sec.~\ref{sec:optimizations}).

\subsection{Backends}

Relay produces machine-specific code
    by decomposing the problem of code generation into multiple distinct phases.
Relay translates all operators into \tvm expressions
    to produce dense linear algebra kernels~\citep{tvm_osdi18, tensor_comprehensions, halide}.
\tvm produces low-level operators that expect a fixed calling convention,
    as well as preallocated inputs and outputs.
The result is an object file containing hardware-specific implementations of all
    operations.
The remaining Relay program then is executed or compiled,
    with operator invocations replaced by calls to the optimized operators.
By representing operators as \tvm expressions, we can programmatically
    transform them and automatically generate new implementations for the transformed operators.
Optimizations like fusion and quantization
    rely on this novel behavior.
After primitive operators are lowered,
    the remaining Relay program ties
    together operator invocations, allocation, control-flow,
    recursion, and high-level data structures.
There are multiple options for executing the combined full program:
    the Relay interpreter (with JIT compilation),
    an Relay virtual machine,
    the \tvm graph runtime,
    and an experimental Relay ahead-of-time compiler
    that converts programs to C++ to produce a target-specific binary.
% We are able to target CPU, GPU,
%   iOS and Android mobile devices,
%   custom accelerators, and FPGAs.

% % EVAL

% We evaluate Relay on several systems and over a diverse set of vision and NLP workloads to
%   demonstrate that (1) Relay enables \emph{expressive} programs via a large breadth
%   of models, (2) Relay supports \emph{composition} of program-level optimizations
%   such as quantization and fusion, and (3) Relay provides
%   \emph{portability} by targeting a number of hardware backends.
% Not only does Relay provide these three properties, we do so while also demonstrating
%   competitive performance.
% Relay is an open-source academic project.\footnote{Relay is publicly available at [redacted for review].}
%   It has been deployed at a popular web service provider,
%     a telecommunications and consumer electronics manufacturer,
%     and a social media company, among others.
%     Our evaluation demonstrates Relay's competitive performance for a
%     broad class of models and devices
%     (CPUs, GPUs, and emerging accelerators).
%   Relay's design demonstrates how a unified IR can provide
%     expressivity, composability, and portability
%     without compromising performance.


%   We evaluate Relay on several systems and over a diverse set of vision and NLP workloads to
%   demonstrate that (1) Relay enables \emph{expressive} programs via a large breadth
%   of models, (2) Relay supports \emph{composition} of program-level optimizations
%   such as quantization and fusion, and (3) Relay provides
%   \emph{portability} by targeting a number of hardware backends.
% Not only does Relay provide these three properties, we do so while also demonstrating
%   competitive performance.
% Relay is an open-source academic project.\footnote{Relay is publicly available at [redacted for review].}
%   It has been deployed at a popular web service provider,
%     a telecommunications and consumer electronics manufacturer,
%     and a social media company, among others.

\subsection{Supporting Hardware Accelerators}
\label{sec:accel}

BYOC

\subsection{VTA}

Hardware specialization is a powerful way to accelerate
  a known set of applications and workloads.
A component of Relay is lowering high-level programs down
  to the bespoke semantics of emerging hardware accelerators.
Unfortunately, deep learning (DL) is anything but a static field, and the machine learning (ML) community
  rapidly changes how they use to write models, the architecture of models themselves, the operators
  used by said models, and the data types they operate over.
Initial programmable accelerators~\citep{tpuv1} offer potentially huge performance
  improvements at the cost of complex specialized compilation.
Furthermore the churn of machine learning has lead to an interest
  in customizable designs, with features such as new numeric representations,
  new hardware engines, and more.
In order to customize the behavior of accelerators designs, even when open-sourced,
  there is a need for the availability of a transparent and modular software stack.
An end-to-end approach requires integration of frameworks, systems, compilers,
  and architecture in order to execute state-of-the-art ML using hardware acceleration.
Peak FLOPs provide value only if a programmer can access them.
In order to tackle this problem I have collaborated on the design for \vta (Versatile Tensor Accelerator),
  an explicitly programmed accelerator paired with a compiler and runtime that can evolve
  in tandem with deep learning models without sacrificing the advantages of specialization.

\vta makes following contributions:

\begin{itemize}
    \item \emph{A programmable accelerator design} that exposes a two-level programming interface: a high-level task ISA to allow explicit task scheduling by the compiler stack, and a low-level microcode ISA to provide software-defined operational flexibility.
    In addition, the \vta architecture is fully parameterizable: the hardware intrinsics, memories, and data types can be customized to adapt the hardware backend requirements.
    \item \emph{An extensible runtime system} for heterogeneous execution that performs JIT compilation of microcoded kernels to provide operational flexibility. For example, the \vta runtime lets us extend the functionality of \vta's original computer-vision-centric design to support operators found in style transfer applications without requiring any hardware modifications.
    \item \emph{A schedule auto-tuning platform} that optimizes data access and reuse in order to rapidly adapt to changes to the underlying hardware and to workload diversity.
\end{itemize}

We recently published a paper on VTA in IEEE Micro Journal Special Issue
  on Deep Learning Acceleration.
\vta is a powerful part of doing research, and Relay is a critical part of the research agenda
  being pursued in the SAMPL lab, with recent funding to explore directly mapping Relay programs
  on to \vta as well as a diverse set of accelerators, and is related to the final topic of my
  thesis automatic tensorization.
