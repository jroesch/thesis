\begin{figure}[htb!]
    \begin{tabular}{ccc}
    \begin{minipage}{0.5\textwidth}
    \begin{minted}[fontsize=\small]{python}
  i = tf.constant(1)
  j = tf.constant(1)
  k = tf.constant(5)

  def c(i, j, k):
    return
      tf.equal(
        tf.not_equal(
          tf.less(i + j, 10),
          tf.less(j * k, 100)),
         tf.greater_equal(k, i + j))

  def b(i, j, k): return [i+j, j+k, k+1]

  tf.while_loop(c, b, loop_vars=[i, j, k])
    \end{minted}
    \end{minipage}
  & \hspace{-3.0em}
  \begin{Huge}
    $\Rightarrow$
  \end{Huge}
  &
    \begin{minipage}{0.5\textwidth}
    \begin{minted}[fontsize=\footnotesize]{python}
  let %while_loop =
    fn (%loop_var0: Tensor[(1,), int32],
        %loop_var1: Tensor[(1,), int32],
        %loop_var2: Tensor[(1,), int32]) {
      %0 = add(%loop_var0, %loop_var1)
      %1 = less(%0, meta[Constant][0])
      %2 = multiply(%loop_var1, %loop_var2)
      %3 = less(%2, meta[Constant][1])
      %4 = not_equal(%1, %3)
      %5 = add(%loop_var0, %loop_var1)
      %6 = greater_equal(%loop_var2, %5)
      if (min(equal(%4, %6))) {
        %9 = add(%loop_var0, %loop_var1)
        %10 = add(%loop_var1, %loop_var2)
        %11 = add(%loop_var2, meta[Constant][2])
        %while_loop(%9, %10, %11)
      } else {
        (%loop_var0, %loop_var1, %loop_var2)
      }
    }
  %while_loop(meta[Constant][3],
              meta[Constant][4],
              meta[Constant][5])
    \end{minted}
    \end{minipage}
    \end{tabular}
    \caption{
      A simple TensorFlow loop in the user-facing DSL and the Relay
        loop produced by automatically converting it.
      Note the TensorFlow while loop corresponds neatly to a tail recursive
        function.
      The Relay text format supports a ``metadata'' section which functions
        as a constant pool among other things.
      \texttt{meta[Constant][n]} represents the \texttt{n}-th constant in the
        pool.
    }
    \label{fig:tf_to_relay_loop}
    \end{figure}
