\chapter{Relay: An IR for deep learning}
\label{ch:relay}

Popular DL compiler intermediate representations (IRs) offer different tradeoffs
between expressivity, composability, and portability~\citep{
  tensorflow, pytorch_ad, chainer_learningsys2015, tangent, theano, glow}.
Early frameworks adopted IRs
  specialized for then-state-of-the-art models and/or
  emerging hardware accelerators.
As a result, non-trivial extensions require
  patching or even forking frameworks~\citep{
    tf_fold, tf_lite, tangent, tf_eager, xla, glow, torchscript}.
Such \textit{ad hoc} extensions can improve expressivity
  while maintaining backwards compatibility with existing execution mechanisms.
However, they are difficult to design, reason about, and implement,
  often resulting in modifications that are mutually incompatible.
Nearly all popular deep learning representations were designed for
  static computation graphs, leading to numerous
  extensions designed to support dynamic neural networks.

This thesis present Relay,
  a new compiler IR for deep learning.
Relay's functional, statically typed intermediate representation (IR)
  unifies and generalizes existing DL IRs
  to express state-of-the-art models.
The introduction of Relay's expressive IR requires
  careful design of domain-specific optimizations,
  addressed via Relay's extension mechanisms.
Using these extension mechanisms,
  Relay supports a unified compiler that
  can target a variety of hardware platforms.
Relay's extensible compiler
   can eliminate abstraction overhead and
   target new hardware platforms.

Previous IRs have struggled to address these challenges, treating each
  component of the framework as a disconnected set of programming tasks.
Operators are defined in low-level languages like C++,
  connected by a dataflow graph, and then scripted
  in a host language like Python.
Consequently
  program analyses cannot cross language boundaries between components,
  inhibiting optimization and deployment.
Relay's design takes inspiration from traditional
  compiler literature where many of the challenges facing
  machine learning compilers have been well studied in the scalar setting.
We analyzed previous deep learning IRs finding ways
  to obtain the desirable properties of these IRs
  with a principled approaches, for example using references
  to split pure and in-pure fragments, or the use of closures
  to represent complex control operators.

Relay is also designed to abstract over platform specific behaviors but not prevent
  representing or optimizing for them.
Given a known target, a user can schedule a new optimization,
  and all necessary platform optimizations and code generation will occur.
Target independence might seem like a property already enjoyed widely,
  but in many frameworks each operator is implemented per platform,
  and often models only work on a single well-supported platform (i.e Nvidia GPU).
Previous IRs are either designed to be tethered to a specific end-user programming model
    or low-level operator library which enables the programs to be executed on specific platforms such as GPU.
Leveraging these features leads to powerful use cases,
  for example we are able to easily get best in class performance on many devices by mixing and matching TVM,
  with native kernel libraries to obtain the best performance, without the end user needing to adapt their program
  in anyway.
The rest of this chapter describes Relay’s IR and type system design, and presents some preliminary
  performance results.
We focus on the following contributions:
\begin{itemize}
  \item The Relay IR, a tensor-oriented, statically typed
    functional IR,
    which we describe in this chapter.
  Relay's design is motivated by the insight that functional IRs, used by
  languages from the ML family\footnote{``ML'' as in ``Meta Language,'' not
  ``Machine Learning''} can be readily adapted to support DL.
  Collections of \textit{ad hoc} extensions in previous frameworks
    that patched shortcomings in expressiveness are subsumed by a handful of well-known language
    constructs like let expressions, ADTs, first-class functions, and references.
  In addition to improving expressivity,
    incorporating these features as language constructs
    allows optimizations to more readily compose.
  \item
  By representing DL models as functional programs, we reframe traditional
    deep learning framework features as compiler problems.
  Backpropagation becomes a source code transformation,
    transforming an arbitrary Relay function into its gradient function;
    \textit{ad hoc} shape inference becomes principled type inference;
    graph rewriting becomes program optimization;
    and the executor becomes (depending on what the context demands) an
    interpreter, virtual machine, or ahead-of-time compiler.
  By using this reframing we can tap into
    decades of traditional compilers research to design
    \textit{composable} optimization passes.
  \item
    A platform-agnostic representation of operators and domain specific
      optimizations which work in concert to provide \textit{portability}
      across hardware backends.
\end{itemize}

\section{IR}

% \begin{figure}[!t]
%     \begin{jmpgrammar}
%       \bnfrule{REAL}{\real} \is{\mathbb{R}}\\
%       \bnfrule{NAT}{\nat} \is{\mathbb{N}}\\
%       \bnfrule{NAME}{\rName} \is{\texttt{(}\text{`\_'}\inlineAlt[a-zA-Z]\texttt{)}\ \
%       \atLeastZero{\text{`\_'}\inlineAlt[a-zA-Z]\inlineAlt[0-9]}}\\
%       \bnfrule{TYPE NAME}{\typename} \is{[A-Z]\ \ \atLeastZero{\text{`\_'}\inlineAlt[a-zA-Z]\inlineAlt[0-9]}}\\
%       \bnfrule{GLOBAL VAR}{\gvar} \is{\kwd{@}\rName}\\
%       \bnfrule{LOCAL VAR}{\lvar} \is{\kwd{\%} \rName}\\
%       \bnfrule{GRAPH VAR}{\graphVar} \is{\kwd{\%} \nat}\\
%       \bnfrule{TYPE VAR}{\tyvar} \is{\rName}\\
%       \bnfrule{OPERATOR}{\op} \is{\rName}\\\\
%       \bnfrule{Program}{\program} \is{\atLeastOne{\defn\inlineAlt\typedef}}
%         \alt{\expr}\\\\
%       \bnfrule{Param}{\param} \is{\rParamRule}\\
%       \bnfrule{Type Param}{\tyParam} \is{\rTyParamRule}\\\\
%       \bnfrule{Definition}{\defn} \is{\rDefnRule}\\\\
%       \bnfrule{Type Definition}{\typedef} \is{\rTypeDefRule}\\\\
%       \bnfrule{Kind}{\kind} \is{\kwd{BaseType}}
%         \alt{\kwd{Shape}}
%         \alt{\kwd{Relation}}
%         \alt{\kwd{ADT}}
%         \alt{\kwd{Type}}\\\\
%       \bnfrule{BaseType}{\basetype} \is{\kwd{int} \nat \maybe{\kwd{x} \nat}}
%         \alt{\kwd{float} \nat \maybe{\kwd{x} \nat}}
%         \alt{\kwd{bool} \maybe{\nat}}\\\\
%       \bnfrule{Shape}{\shape} \is{\kwd{(} \seq{\nat} \kwd{)}}\\\\
%       \bnfrule{Pattern}{\patt} \is[constructor]{\op \kwd{(} \seq{\patt} \kwd{)}}
%         \alt[wildcard]{\wildcard}
%         \alt[variable]{\lvar\ \maybe{\typeanno}}
%     \end{jmpgrammar}
%   \end{figure}
  \begin{figure}[t]
    % \ContinuedFloat
    \begin{jmpgrammar}
      \bnfrule{Expr}{\expr} \is[local var]{\lvar}
        \alt[global variable]{\gvar}
        \alt[constant tensor]{\kwd{const} \kwd{(} \texttt{(}\real\inlineAlt\bool\texttt{)} \kwd{,} \shape \kwd{,} \basetype \kwd{)}}
        \alt[call]{\expr \maybe{\tyargs} \args\vspace{0.2em}}
        \alt[let]{\rLetRule}
        \alt[\kwd{let}\ \kwd{\%}\_\ \kwd{=}\ \expr\kwd{;}\ \expr]{\rAnonLetRule}
        \alt[graph let]{\rGraphLetRule}
        \altSpace{0.5em}{function}{\rFnRule}
        % \altSpace{1em}{type cast?}{\kwd{(} \type \kwd{)} \expr}
        \altSpace{1em}{tuple formation}{\rTupRule}
        \alt[tuple proj.]{\rTupProjRule}
        \alt[if-else]{\rIfElse{\expr}{\expr}{\expr}}
        \altSpace{0.5em}{pattern match}{\rMatchRule}
        \altSpace{1em}{operator}{\op}
        % TODO: When will we have gradient as a first-class operator?
        % \alt[gradient]{\kwd{grad}\kwd{(}\expr\kwd{)}}
        \alt[new ref]{\kwd{ref}\kwd{(}\expr\kwd{)}}
        \alt[get ref]{\kwd{!} \expr}
        \alt[set ref]{\expr \kwd{:=} \expr}\\\\
      \bnfrule{Type}{\type} \is[base type]{\basetype}
        \alt[shape]{\shape}
        \alt[tensor type]{\kwd{Tensor} \kwd{[} \shape \kwd{,} \basetype \kwd{]}}
        \alt[type variable]{\tyvar}
        \alt[function type]{
          \begin{split}
          \kwd{fn}\ &\rTyParamsRule\\
          &\kwd{(} \seq{\type} \kwd{)}\ \rettype\\
          &\maybe{\relations}
          \end{split}
          }
        \alt[ref type]{\kwd{Ref} \kwd{[} \type \kwd{]}}
        \alt[tuple type]{\kwd{(} \seq{\type} \kwd{)}}
        \alt[type call]{\type \kwd{[} \seq{\type} \kwd{]}}
        \alt[type name]{\typename}
    \end{jmpgrammar}
    \caption{\textmd{The BNF Grammar for the \relay{} language.}}
    \label{fig:short_bnf}
  \end{figure}


The Relay IR is a high-level, functional, differentiable language.
Relay is designed to support
  complex models while abstracting over hardware-specific
  implementation details to enable hardware agnostic program
  analysis and optimization.
Rather than invent an entirely new language,
  Relay's IR design is based on IRs used by the well-studied ML family of
  functional programming languages (e.g., SML and OCaml).
These IRs are expressive enough to capture general-purpose programs
  (including control flow, first-class functions, and data types)
  and have clearly specified semantics (e.g., lexical scope and controlled effects).
By borrowing from PL literature,
  we can apply program analysis and optimization techniques from decades of research~\citep{haskell_vector}.
Relay's IR takes a small functional core and enriches it with domain-specific additions---namely,
  the inclusion of tensors and operators as expressions
  and a novel tensor type system design to support tensor shapes.
Our principled design
  enables the import of existing models from deep learning frameworks and exchange formats,
  the implementation of a number of domain-specific optimizations,
  and efficient deployment across a variety of targets.
The Relay IR is designed
  to subsume the functionality of computation graph-based IRs
  while providing greater faculties for abstraction, and dynamism.
We present Relay's design by incrementally building up to the full IR
  starting from a subset that corresponds to a simple computation graph.

Hence, Relay's primary value type is a tensor and operators are included as language primitives
  (see the \verb|tensor constant| and \verb|operator| rules in Figure \ref{fig:short_bnf}).
Relay leaves the implementation of each operator opaque; the operators
  are represented by a lower-level IR, which is optimized independently.
A computation graph, in its simplest presentation, is a directed acyclic
  graph with multiple inputs and a single output.
The syntax of an equivalent computation graph is realized by
  a language with three rules (1) \verb|variable|s, (2) function \verb|call|s,
  and (3) \verb|operator|s, see Figure~\ref{fig:short_bnf} for the corresponding rules.

\subsection{Operators}

Operators are the primitive computation of Relay and represented using
  an opaque function signature, and backed by fragments of TVM's TE or TIR
  which can be lowered to low-level device specific code to realize
  the abstract operations with specialized implementations.

\subsection{Multiple Outputs}

Many common operators like \verb|split|, which splits
  a tensor along a particular axis, require multiple outputs.
In order to handle these programs,
  computation graph IRs have added primitive support
  for multiple outputs.
Multiple outputs can be modeled as tuples, which can
  be added with just two rules (1) \verb|tuple formation|
  and (2) \verb|tuple projection|.
Instead of using multiple outputs as a builtin concept the use
  of tuples allow us to represent operations which not only return
  one level of data structure, but can also use nested structures
  such as tuples of tuples.

\subsection{Let}

By construction, computation graphs enjoy implicit sharing of subcomputations
  via multiple outgoing dependency edges.
Implicit sharing is often implemented via pointers that uniquely identify subgraphs,
  a property useful for both execution and analysis.
Previous frameworks often obtain this sharing by using a host
  language's name binding to construct a graph (e.g., by binding a Python variable
  to a subgraph and using that variable to construct other subgraphs).
General-purpose programming languages, on the other hand, provide \textit{explicit}
  sharing via binding constructs, such as \verb|let|.
In programs free of scope, ordering, and effects, implicit sharing
  and explicit sharing are semantically equivalent.
However, in practice, user programs rely on effects and ordering,
  requiring previous approaches to provide workarounds.
For example, TensorFlow's Eager Mode inserts dummy control edges
  in its generated graphs to impose effect ordering.
The lack of lexical scope in computation graphs complicates language features,
  like first-class functions and control flow,
  and reduces the precision of traditional analyses,
  such as liveness,
  because the high-level program structure is absent~\citep{funarg, funarg_sol}.
The addition of a humble \verb|let| binding, a central concept in functional languages,
  provides explicit sharing and a solution to the problems outlined above.

\subsection{Control Flow}

Emerging models, particularly in the domain of natural language processing,
  increasingly rely on data-dependent control flow,
  forcing frameworks based on computation graph IRs
  to incorporate control flow, often through \textit{ad hoc} and difficult-to-extend constructs.
For example, TensorFlow Fold~\citep{tf_fold} extends TF with special combinators that
  dynamically compute a graph for each shape permutation;
  these high-level constructs are opaque to further optimizations.
Even in the presence of control flow-free models, looping
  constructs are necessary to implement optimization algorithms
  such as SGD.
The central challenge is a flexible and extensible encoding of
  control flow operations.
It is well known in functional programming literature that recursion and pattern matching are sufficient
  to implement arbitrary combinators for control flow and iteration (e.g., maps, folds, and scans).
To support the definition of functional combinators
  we enrich Relay with two more language
  features to implement arbitrary combinators: \verb|if| and first-class recursive functions.

\subsection{First-Class Functions}

A computation graph is a single expression
  from multiple inputs (i.e. its free variables) to multiple outputs.
While it may be tempting to reinterpret a graph as a function, it lacks functional abstraction
  and named recursion.
Adding the ability to name functions and pass them as first-class values dramatically increases
  Relay's expressivity, allowing it to encode generic
  higher-order functions and readily use techniques used in functional
  compilers like automatic deforestation.
First-class functions enable passes such as
  automatic differentiation, and simplify
  the framework importers which map higher-level programs to our IR~\citep{myia}.
For example, an instance of TensorFlow's looping construct \verb|tf.while_loop|
  can be represented as a single specialized loop function
  or a generic fold over the full loop state.
See Figure~\ref{fig:tf_to_relay_loop} for an example of this conversion (via
  the Relay TensorFlow frontend).

\subsection{Data Abstraction}
Many models make use of additional data types beyond
  tuples, such as lists, trees, and graphs~\citep{char-rnn, tree_lstm, graph_lstm}.
Relay borrows from functional languages
  a generic and principled method of extension:
  algebraic data types (ADTs).
To support them, we add mechanisms for
  (1) type declaration and
  (2) pattern matching.
This final addition results in a strict functional language,
  closely resembling the core of languages like OCaml and SML.
The increase in expressivity introduced by the Relay IR introduces
  new optimizations challenges, which we
  discuss in Chapter~\ref{ch:related}.

% --------------------------------------------------------------------------------------------------------
The resulting language is a familiar strict functional language,
  resembling the core of languages like OCaml and SML.
Our language makes domain-specific deviations from existing work,
  and we have provided a full listing
  of its syntax, operational semantics, and type rules
  in the appendix.
A functional language provides a few notable advantages.
Its pure fragment represents idealized computation graphs free
  from effects. This fragment can be easily optimized by end users who
  can reason about it as pure dataflow.
For this reason, Relay is pure by default but exposes a limited
  form of mutation via ML-style references that we have
  primarily used for automatic differentiation.

Relay is more expressive than many previous frameworks and this expressivity introduces new challenges.
  Previous essential functionality such
    as shape inference and automatic differentiation must be adapted for
    our new IR.
How does one reason about the shapes of operators when the input is unknown?
How does one backpropagate over pattern-matching, control, data types, and mutation?
In the following subsection we demonstrate how one can adapt techniques
  from type inference and checking to Relay.

% Another important semantic choice is that all Relay operators expose a functional interface, i.e they consume inputs and outputs.
% The functional semantics allow us to later introduce mutable buffers without requiring complex interprocedural program analysis.
% Our second principle was a transparent separation from the abstract representation of machine learning models, and kernels.
% The abstraction is successfully introduced by computation graph IRs, but in an opaque way which limits cross layer optimization.
% Many gestalt approaches use a single language that defines all kernels and models, requiring users to both reason about all levels during optimization.

% We are able to represent abstract operations with well known types at the Relay level via the introduction of a user extensible shape dependent
%   type system which can represent a wide range of operator’s types.

% Our final guiding principle has been target independence.
\begin{figure}[htb!]
    \begin{tabular}{ccc}
    \begin{minipage}{0.5\textwidth}
    \begin{minted}[fontsize=\small]{python}
  i = tf.constant(1)
  j = tf.constant(1)
  k = tf.constant(5)

  def c(i, j, k):
    return
      tf.equal(
        tf.not_equal(
          tf.less(i + j, 10),
          tf.less(j * k, 100)),
         tf.greater_equal(k, i + j))

  def b(i, j, k): return [i+j, j+k, k+1]

  tf.while_loop(c, b, loop_vars=[i, j, k])
    \end{minted}
    \end{minipage}
  & \hspace{-3.0em}
  \begin{Huge}
    $\Rightarrow$
  \end{Huge}
  &
    \begin{minipage}{0.5\textwidth}
    \begin{minted}[fontsize=\footnotesize]{python}
  let %while_loop =
    fn (%loop_var0: Tensor[(1,), int32],
        %loop_var1: Tensor[(1,), int32],
        %loop_var2: Tensor[(1,), int32]) {
      %0 = add(%loop_var0, %loop_var1)
      %1 = less(%0, meta[Constant][0])
      %2 = multiply(%loop_var1, %loop_var2)
      %3 = less(%2, meta[Constant][1])
      %4 = not_equal(%1, %3)
      %5 = add(%loop_var0, %loop_var1)
      %6 = greater_equal(%loop_var2, %5)
      if (min(equal(%4, %6))) {
        %9 = add(%loop_var0, %loop_var1)
        %10 = add(%loop_var1, %loop_var2)
        %11 = add(%loop_var2, meta[Constant][2])
        %while_loop(%9, %10, %11)
      } else {
        (%loop_var0, %loop_var1, %loop_var2)
      }
    }
  %while_loop(meta[Constant][3],
              meta[Constant][4],
              meta[Constant][5])
    \end{minted}
    \end{minipage}
    \end{tabular}
    \caption{
      A simple TensorFlow loop in the user-facing DSL and the Relay
        loop produced by automatically converting it.
      Note the TensorFlow while loop corresponds neatly to a tail recursive
        function.
      The Relay text format supports a ``metadata'' section which functions
        as a constant pool among other things.
      \texttt{meta[Constant][n]} represents the \texttt{n}-th constant in the
        pool.
    }
    \label{fig:tf_to_relay_loop}
    \end{figure}


\subsection{Type System}
  \label{subsec:type_system}

  Relay's type system is essential
    to optimizations.
  Typing guarantees both well-formedness of the program
    and provides crucial tensor shape information to perform allocation,
    check correctness, and facilitate loop optimizations.
  Shape information is also valuable for data layout transformations and tensorization,
    two transformations often demanded by hardware accelerators.
  In computation graph IRs, only numeric data types
    and shapes are tracked for each operator.
  Symbolic shapes (i.e., shape polymorphism) are only handled
    dynamically, inhibiting certain types of optimizations.

  It is possible to model arbitrarily complex static properties, such
    as shape information, with a dependent type theory~\citep{selsam_certigrad}, but such
    a design incurs significant user complexity.
  By incorporating shape analysis into a broader type system,
    Relay's type system balances the desire for static tensor shapes
    with usability.
  In this subsection, we describe how to extend a polymorphic type system with shape
    information and type inference with shape inference.

  \subsection*{Tensor Types}

  The primitive value in Relay is a tensor, which has
    a shape and a base type (\verb|tensor type| in Figure \ref{fig:short_bnf}).
  Base types describe the elements of tensors by tracking
    the bit width,
    the number of lanes (for utilizing vectorized intrinsics),
    and whether the type is floating point or integral.
  To ensure Relay can offload tensor computation to devices
    with greatly varying architectures,
    Relay tensors may only contain base types,
    preventing, for example, tensors of closures.
  The shape of a tensor is a tuple of integers describing the tensor's dimensions.
  A dimension may be a variable or arithmetic expression that indicates how the
    output shape of an operator depends on those of its inputs.
  Functions may be polymorphic over shapes, which results
    in shape constraints that must be solved during type inference.
  Sec.~\ref{sec:inference} describes the process.
  Relay also supports a special shape called \verb|Any|, which is used
    to mark a dynamic shape when static relationships are not profitable
    to model.

  \subsection*{Operators and Type Relations}
  Operators are one of the key primitives that differs from those of
    general-purpose programming languages.
  Relay's use of opaque operators enables backends to choose different
    lowering strategies based on the hardware target.
  Relay's operator set is extensible, meaning that users may add new operations.
  Supporting common or user-defined tensor operators requires a type system that can
    adapt to complex shape relationships between input and output types
    (e.g., elementwise operators with broadcasting semantics).

  To handle the constraints between operators' argument shapes, Relay's type system
    introduces type relations.
  A type relation is implemented as a function in the
    meta-language and represents a symbolic relationship between
    the input and output types.
  When developers add a new operator to Relay, they may constrain its
    type with an existing relation or add their own.
  Function types may include
    one or more type relations over a subset of the argument types and the return type.
  The type checker enforces that these relationships hold at each call site.

  \section{Type Inference}
  \label{sec:inference}

  The most interesting parts of the type system
    are where shape computation occurs.
  We highlight a few examples of Relay's inference
    rules in Fig.~\ref{fig:partial-inference-rules};
    the full typing rules can be found in the appendix.
  In this subsection we focus on design decisions behind Relay's type system
    and the implementation of type inference.
  To incorporate type relations into Relay's type system, we enrich
    a Hindley-Milner type inference algorithm with
    a constraint solver for type relations.
  Relay's inference algorithm has three steps: first it
    performs a pass over the AST generating types (potentially involving type variables)
    as well as populating the set of relations,
    then it solves the incurred constraints,
    and finally it assigns types to each expression in the AST.
  A type relation is implemented as a function in the
    meta-language and represents the symbolic relations between
    the input and output types of an object-language function.
  When the type inference algorithm visits a function call site, the function's type relations are
    instantiated to the types at the call site and added to a queue of relations waiting to be
    solved.
  The relationships between the call's type variables and its relations are are added to a
    bipartite undirected dependency graph where the two disjoint sets are type variables and type relations.
  Traditional unification constraints are represented using a modified union-find structure that
    integrates with the dependency graph.

  Once the queue is populated, the algorithm will dequeue a relation and attempt to solve it.
  There are two cases when solving a type relation:
  \begin{enumerate}
    \item If all the relation's type variables
    are concrete, we call the type relation function. If that function returns true, the
    constraint is discharged. Otherwise, type checking fails.
    \item If any type is fully or partially symbolic, the
      algorithm will propagate
      existing concrete type information via unification.
    All relations affected by new assignments to type
      variables (as determined by the dependency graph)
      are moved to the beginning of the queue.
    If the current type relation is now completely solved, we
    discard it to avoid unnecessarily visiting it again.
  \end{enumerate}

  Our fine-grained dependence graph provides the transitive dependencies
    between relations and unification variables.
  The use of fine-grained dependencies enables our algorithm to
    only retry a minimal number of relations when we
    learn a new variable assignment.
  We run this to fixpoint or until the queue is empty.
  If the queue is not empty and no progress is made between iterations,
    then at least one variable is under constrained and inference fails.
  Note that a type relation's implementation can
    compromise type soundness, as they are axiomatic descriptions
    of operations implemented outside of Relay.
  Luckily, in practice, the number of type relations needed to express most of Relay's
    operators is relatively small, and their implementations are generally straightforward
    and amenable to exhaustive testing.
The above sections discuss the core of Relay, a full-length paper under submission\citep{roesch2019relay}
  details its design and implementation.

  To incorporate type relations into Relay's type system, we enrich
    a Hindley-Milner-style type inference algorithm with
    a constraint solver.
  Relay's inference algorithm has three steps: first, it
    performs a pass over the AST, generating types and a set of relations,
    then it solves the incurred constraints,
    and finally annotates each sub-expression with its inferred type.

  % a queue of relations. dequeue a relation and attempt to solve it by invoking it's opaque
  % type-relation function.
  % 1. all type variables concrete & relationship holds => solved. discard relation
  % 2. all type variables concrete & relationship doesn't hold => type error
  % 3. types are partially symbolic => propagate unif. constraints to its inputs and output. if
  % all type variables are solved/assigned/concrete, discard relation. otherwise add it and the
  % relations that depend on the solved variables to queue. those that are already on the queue should
  % be reprioritized

  When the type inference algorithm visits a function call site, the function's type relations are
    instantiated with the concrete argument types at the call site.
  Each instantiated relation is added to the queue of relations to solve.
  The relationship between a call's type variables and relations is added as an edge to
    a bipartite dependency graph where the two disjoint sets are type variables and type relations.
  Traditional unification constraints are represented using a modified union-find structure that
    integrates with this dependency graph.

  Once the queue is populated, the algorithm will dequeue a relation and attempt to solve it.
  There are two cases when solving a type relation:
  \begin{enumerate}
    \item If all the relation's type variables
    are concrete, we the relation function. If that function returns true, the
    constraint is discharged. Otherwise, type checking fails.
    \item If any type is fully or partially symbolic, the
      algorithm will propagate
      existing concrete type information via unification.
    All relations affected by new assignments to type
      variables (as determined by the dependency graph)
      are moved to the beginning of the queue.
    If the current type relation is now completely solved, we
    discard it to avoid unnecessarily visiting it again.
  \end{enumerate}

  We run this to fixpoint or until the queue is empty.
  If the queue is non-empty and no progress is made between iterations,
    then at least one variable is underconstrained and inference fails.
  Note that a type relation's implementation can
    compromise type soundness, as they are axiomatic descriptions
    of operations implemented outside of Relay.
  In practice, the number of type relations needed to express Relay's
    operators is small, and their implementations are straightforward
    and amenable to exhaustive testing.

    To incorporate type relations into Relay's type system, we enrich
      a Hindley-Milner-style type inference algorithm with
      a constraint solver.
    Relay's inference algorithm has three steps: first, it
      performs a pass over the AST, generating types and a set of relations,
      then it solves the incurred constraints,
      and finally annotates each sub-expression with its inferred type.

    % a queue of relations. dequeue a relation and attempt to solve it by invoking it's opaque
    % type-relation function.
    % 1. all type variables concrete & relationship holds => solved. discard relation
    % 2. all type variables concrete & relationship doesn't hold => type error
    % 3. types are partially symbolic => propagate unif. constraints to its inputs and output. if
    % all type variables are solved/assigned/concrete, discard relation. otherwise add it and the
    % relations that depend on the solved variables to queue. those that are already on the queue should
    % be reprioritized

    When the type inference algorithm visits a function call site, the function's type relations are
      instantiated with the concrete argument types at the call site.
    Each instantiated relation is added to the queue of relations to solve.
    The relationship between a call's type variables and relations is added as an edge to
      a bipartite dependency graph where the two disjoint sets are type variables and type relations.
    Traditional unification constraints are represented using a modified union-find structure that
      integrates with this dependency graph.

    Once the queue is populated, the algorithm will dequeue a relation and attempt to solve it.
    There are two cases when solving a type relation:
    \begin{enumerate}
      \item If all the relation's type variables
      are concrete, we the relation function. If that function returns true, the
      constraint is discharged. Otherwise, type checking fails.
      \item If any type is fully or partially symbolic, the
        algorithm will propagate
        existing concrete type information via unification.
      All relations affected by new assignments to type
        variables (as determined by the dependency graph)
        are moved to the beginning of the queue.
      If the current type relation is now completely solved, we
      discard it to avoid unnecessarily visiting it again.
    \end{enumerate}

    We run this to fixpoint or until the queue is empty.
    If the queue is non-empty and no progress is made between iterations,
      then at least one variable is underconstrained and inference fails.
    Note that a type relation's implementation can
      compromise type soundness, as they are axiomatic descriptions
      of operations implemented outside of Relay.
    In practice, the number of type relations needed to express Relay's
      operators is small, and their implementations are straightforward
      and amenable to exhaustive testing.

      \subsection{Type System}
      \label{subsec:type_system}

      Computation graph IRs rely on typing in the form of
        datatype and shape inference.
      Datatype and shape inference is the process of computing the
        concrete datatypes (e.g., \verb|float32|, \verb|int32|) and shapes (e.g., $(10, 5)$, $(100, 1, 32)$) of all
        tensors in a computation graph.
      Deep learning frameworks and compilers use static shape information
        to perform allocation, check correctness, and facilitate optimization.
      Precise static shape information is also valuable for traditional loop
        optimizations, data layout transformations, tensorization, and
        optimizations that are necessary to map to hardware accelerators' unique ISAs.

      Shape inference is usually formulated as a simple analysis over the dataflow graph that
        propagates shape information.
      Shape inference looks remarkably similar to type inference.
      Unlike type inference, though, shape inference is separate from the type system and
        does not provide types for functions or data structures.
      Handling shape inference at compile time is desirable, because it allows optimizations to take
        advantage of this information even though certain shapes may be symbolic. Can shape information be encoded in static types?
      % If we type Relay as simply typed lambda calculus,
      %   we gain a simple system, but one that can not represent polymorphism,
      %   and lacks shape information.
      % Even with the addition of polymorphism, there is no representation of static
      %   shape information.
      It is possible to model arbitrarily complex static properties, such
        as shape information, with dependent type theory, but such
        a design incurs significant user complexity.
      % Adopting a well known type system allows the application of
      %   classic techniques, but standard type systems do not
      %   provide a solution for tracking static shape information.
      Relay's type system is designed to balance the desire for static tensor shapes
        without limiting the language's expressiveness.
      In this subsection we describe how to extend a polymorphic type system with shape
        information and type inference with shape inference.

      \begin{figure}
        \begin{footnotesize}
          \judgbox{\typecheck{\typeCtx}{\varCtx}{\expr}{\type}}{Expression $\expr$ has type $\type$ in type context $\typeCtx$ and variable context $\varCtx$.}
          \begin{inference}
          \jmpInfer{Relation-T}
             {\Delta, T_1 : \texttt{Type}, \ldots, T_n : \texttt{Type} \vdash (Rel(T_1, T_2, \ldots, T_n) \in \{ \top, \bot \}) }
             {\typecheck{\typeCtx}{\varCtx}{Rel}{\texttt{Relation}}}

          \jmpInfer{Type-Func-Def}
            {\forall{i \in [1, r]} \, \Delta; \Gamma \vdash R_i(T_1, \ldots, T_n, O) \\
             \Delta; \Gamma, a_1 : T_1, \ldots, a_n : T_n, \\
             f : \kwd{fn}( T_1, \ldots, T_n) \rightarrow O \kwd{ where } R_1, \ldots, R_r \vdash body : O}
              {\Delta; \Gamma \vdash \kwd{def @} f\kwd{(}a_1\kwd{:} T_1\kwd{,} \ldots
              a_n\kwd{:} T_n\kwd{)} \rightarrow O \kwd{ where } R_1, \ldots, R_r \kwd{ \{ } body \kwd{ \}} : \\
              \kwd{fn}(T_1, \ldots, T_n) \rightarrow O \kwd{ where } R_1, \ldots, R_r }

          \jmpInfer{Type-Call}
             {\Delta; \Gamma \vdash f : \kwd{fn} (T_1 \kwd{,} \ldots \kwd{,} T_n) \rightarrow O
               \kwd{ where } R_1, \ldots, R_r
               \\ \Delta; \Gamma \vdash a_1 : T_1, \ldots, a_n : T_n
               \\ \forall{i \in [1, r]} \, \Delta; \Gamma \vdash R_i(T_1, \ldots, T_n, O)}
             {\typecheck{\typeCtx}{\varCtx}{f(a_1, \ldots, a_n)}{O}}
          \end{inference}
        \end{footnotesize}
        \caption{Examples of Relay's typing inference rules, namely the rules for function definitions and function calls,
          where $\Delta$ is the environment for types and $\Gamma$ is the environment for variables. These demonstrate
          that type relations must hold at each call site.}
        \label{fig:partial-inference-rules}
      \end{figure}

      \subsection{Tensor Types}

      The primitive value in Relay is a tensor, which has
        type $Tensor[s, bt]$ where $s$ is a shape and $bt$ is a base type.
      Elements of base type are floating point numbers and
        integers of specific bit widths and number of lanes.
      This design decision is inspired by LLVM,
        which supports arbitrary-width integer types.
      The parameterization by lanes helps represent vectorized data types, which are supported
        by many CPUs and hardware accelerators.
      To ensure Relay can offload tensor computation to devices
        with greatly varying architectures,
        Relay's kinding rules only permit tensors to contain
        base types, preventing, for example, tensors of closures.

      The shape of a tensor is a tuple of integers describing the tensor's dimensions.
      In general, these dimensions may depend on arguments to an operator.
      A dimension may be a variable or arithmetic expression that indicates how the
        output shape of an operator depends on those of its inputs.
      Functions may be polymorphic over shapes, which results
        in shape constraints that must be solved during type inference.
      Sec.~\ref{sec:inference} describes the process.
      Relay also supports a special shape called \verb|Any|, which is used
        to indicate that we do not have static shape information about a particular dimension.

      \subsection{Operators and Type Relations}

      A difference between general purpose programming models and those tailored to deep learning
        is the use of operators as the primitive unit of computation.
      The ability to add new operations to Relay requires a type system that can adapt to
        complex shape relationships between input and output types.
      Many operators have types that can be defined
        as functions of the input types.
      Unfortunately some are not only functions,
        but also relations that specify constraints between input and output shapes.
      A key extension of Relay over traditional type systems is the addition of type relations
        to express these constraints.
      When developers add a new operator to Relay, they may constrain its
        type with existing relations or add their own.
      Function types (including those of operators) may include
        one or more type relations over an arbitrary subset of the argument types and the return type.
      The type checker enforces that these relationships hold at the call site.
      These relations may be viewed as a verification condition induced at a
        function call site, where the formula is a conjunction of the relations.
      For example, primitive operators are assigned types that are universally quantified over
        both the input and output types.
      We can then use a type relation to encode a constraint that must hold later
        when type checking observes specific input and output types.
      Type relations are opaque in the Relay IR: they are implemented in the
        meta-language and registered when defining an operator.
      However, they may be reused across different implementations.
      For example, we use a relation that describes the
        broadcasting rule for all elementwise operations.

        \subsection{Type System}
        \label{subsec:type_system}

        Relay's type system is essential
          to optimizations.
        Typing guarantees both well-formedness of the program
          and provides crucial tensor shape information to perform allocation,
          check correctness, and facilitate loop optimizations.
        Shape information is also valuable for data layout transformations and tensorization,
          two transformations often demanded by hardware accelerators.
        In computation graph IRs, only numeric data types
          and shapes are tracked for each operator.
        Symbolic shapes (i.e., shape polymorphism) are only handled
          dynamically, inhibiting certain types of optimizations.

        It is possible to model arbitrarily complex static properties, such
          as shape information, with a dependent type theory~\citep{selsam_certigrad}, but such
          a design incurs significant user complexity.
        By incorporating shape analysis into a broader type system,
          Relay's type system balances the desire for static tensor shapes
          with usability.
        In this subsection, we describe how to extend a polymorphic type system with shape
          information and type inference with shape inference.

        \subsection*{Tensor Types}

        The primitive value in Relay is a tensor, which has
          a shape and a base type (\verb|tensor type| in Figure \ref{fig:short_bnf}).
        Base types describe the elements of tensors by tracking
          the bit width,
          the number of lanes (for utilizing vectorized intrinsics),
          and whether the type is floating point or integral.
        To ensure Relay can offload tensor computation to devices
          with greatly varying architectures,
          Relay tensors may only contain base types,
          preventing, for example, tensors of closures.
        The shape of a tensor is a tuple of integers describing the tensor's dimensions.
        A dimension may be a variable or arithmetic expression that indicates how the
          output shape of an operator depends on those of its inputs.
        Functions may be polymorphic over shapes, which results
          in shape constraints that must be solved during type inference.
        Sec.~\ref{sec:inference} describes the process.
        Relay also supports a special shape called \verb|Any|, which is used
          to mark a dynamic shape when static relationships are not profitable
          to model.

        \subsection*{Operators and Type Relations}
        Operators are one of the key primitives that differs from those of
          general-purpose programming languages.
        Relay's use of opaque operators enables backends to choose different
          lowering strategies based on the hardware target.
        Relay's operator set is extensible, meaning that users may add new operations.
        Supporting common or user-defined tensor operators requires a type system that can
          adapt to complex shape relationships between input and output types
          (e.g., elementwise operators with broadcasting semantics).

        To handle the constraints between operators' argument shapes, Relay's type system
          introduces type relations.
        A type relation is implemented as a function in the
          meta-language and represents a symbolic relationship between
          the input and output types.
        When developers add a new operator to Relay, they may constrain its
          type with an existing relation or add their own.
        Function types may include
          one or more type relations over a subset of the argument types and the return type.
        The type checker enforces that these relationships hold at each call site.

\section{Evaluation}
\label{sec:eval}

% 5. Methodology and evaluation: how do you plan to evaluate whether your ideas
% work or your hypotheses are correct?

In my proposal I make three specific claims we must evaluate,
\textbf{To develop an intelligent (1) end-to-end deep learning compiler that can
automatically adapt (2) high-level programs to run on new hardware devices automatically
with (3) human level performance.}

We evaluated Relay on several systems (x86 CPUs, ARM CPUs, NVIDIA GPUs, Xilinx FPGAs) and over
  diverse vision and NLP workloads to demonstrate that (1) Relay enables \emph{composability} of
  graph-level optimizations, (2) Relay delivers \emph{performance} on inference tasks competitive
  with state-of-the-art frameworks (TensorFlow, PyTorch, MxNet), and (3) Relay provides
  \emph{portability} over difficult-to-compile-to hardware backends such as FPGAs

\subsection{Relay: A high-level IR for Deep Learning}
  In particular, our evaluation is composed of three parts:
  \begin{enumerate}
    \item \textbf{Relay enables composable optimizations}: Relay
      supports composing program transformations into multiple optimization tiers.
    \item \textbf{Relay provides competitive performance}: Despite increasing
      expressiveness, Relay's performance is competitive with the
      state of the art on popular models.
    \item \textbf{Relay handles challenging backends}: Relay can compile
      models to execute efficiently on a variety of
      backends, such as FPGA accelerators, which require quantization, layout
      optimizations, and bit-packing transformations.
    % \item \textbf{Relay supports expressive models}: It can compile models with complex control flow
    % like Char-RNN to a single lean binary, beating PyTorch inference performance by up to 3x.
  \end{enumerate}

  We evaluated the following vision models:
    \textit{Deep Q-Network (DQN)}, a DNN that achieved state-of-the-art performance
    on 49 Atari games in 2015;
    \textit{MobileNet}, a DNN designed for image recognition on mobile and
    embedded devices;
    \textit{ResNet-18}, a DNN for image recognition that achieved state-of-the-art
    performance on ImageNet detection tasks in 2015;
    \textit{VGG-16} (named for the Visual Geometry Group
    at Oxford), a CNN used for image recognition tasks
    \citep{dqn, mobilenet, resnet, vgg}.

  We evaluated the following NLP models:
    \textit{CharRNN}, a generator character-level
    RNN from a PyTorch tutorial;
    \textit{TreeLSTM}, a generalization of LSTMs to
    tree-structured network topologies;
    \textit{RNN, GRU, and LSTM}, a selection of models from the Gluon
    model zoo
    \citep{pytorch_rnn_tut, tree_lstm, gluon_model_zoo}.

  \subsection{Experimental Methodology}
  % Because we only evaluate inference in this paper,
  %   we frequently make use of random inputs to models when measuring
  %   performance.
  % There were two exceptions where we evaluated on real data because
  %   it was readily available: CharRNN and TreeLSTM.

  % % For models where the input is random,
  % %   we run 1000 timed iterations.
  % % Before the timed runs,
  % %   we run 8 untimed ``warm-up'' iterations to ensure any caching and JIT compilation
  % %   employed in lower levels of the stack are included in the 1000 timed runs.
  % % This way,st
  % %   the timed runs reflect the \textit{stable} performance of the system.
  % % For our purposes, ``performance'' refers to end-to-end framework time on
  % %   inference tasks (i.e., the time it takes to run a trained model) in a
  % %   single-machine setting.

  Our vision experiments from Chapter~\ref{ch:related} and Section~\ref{sec:perf-gpu} were run on a machine with an AMD Ryzen
    Threadripper 1950X 16-Core CPU,
    an NVidia 1080 Ti GPU,
    and 64 GB of RAM.
  Our NLP experiments from Section~\ref{sec:perf-gpu} were run on a machine with an AMD Ryzen
    Threadripper 1950X 16-Core CPU,
    an NVidia Titan-V GPU,
    and 64 GB of RAM.
  Our low-power vision experiments from Section~\ref{sec:low-power} were run on multiple edge-class ARM development boards: a RaspberryPi 3, a Firefly RK3399, and an Ultra-96 FPGA platform.
  We evaluated Relay's handling of accelerators on a VTA design with a
    $16\times16$ matrix-vector 8-bit tensor core clocked at 333MHz on the Ultra-96 platform.

  % In terms of software, we used
  %   Cuda version 10.0,
  %   CuDNN version 7.5.0,
  %   TVM commit \texttt{cefe07e2a}\footnote{NLP experiments required custom modifications that may be made public later},
  %   MxNet version 1.4.0,
  %   Pytorch version 1.0.1post2,
  %   and TensorFlow version 1.13.1.

  The Relay vision experiments utilized aggressively tuned TVM schedules on the GTX 1080 Ti GPU,
    improving performance significantly.
  % \subsection{Relay Enables Composable Optimizations}
  % \label{sec:optimizations}

  % \begin{figure}[h]
  %   \includegraphics[width=0.5
  %   \textwidth]{fig/eval/optimization_levels.pdf}
  %   \caption{Speedup from increasing the number of graph transformations in Relay (\texttt{-O1}, \texttt{2}, \texttt{3}), relative to no optimizations at all (\texttt{-O0}). We show that, by composing passes, we can monotonically improve performance on vision benchmarks running on the NVIDIA GTX 1080 Ti.}
  %   \label{fig:opt-eval}
  % \end{figure}

  % We demonstrate that Relay can facilitate composable optimizations,
  % by evaluating vision workloads under incremental optimization levels, denoted \texttt{-On}:
  % \begin{itemize}
  %   \item \texttt{-O0} does not apply any program transformation passes.
  % % (it applies a single pass to translate the ``batch norm'' operator into simpler arithmetic
  % % operators but Relay{} has no other implementation of ``batch norm'').
  %   \item \texttt{-O1} applies an operator fusion pass.
  %   \item \texttt{-O2} additionally applies constant folding, using Relay{}'s interpreter to evaluate away operations on constants.
  %   \item \texttt{-O3} additionally applies four more passes:
  %       (1) \texttt{FoldScaleAxis}, which folds scaling operations into the axis options of other operators,
  %       (2) \texttt{AlterOpLayout}, which alternates operator layouts for better cache performance,
  %       (3) \texttt{CanonicalizeOps}, which canonicalizes the ``bias add'' operator in terms of expanding dimensions and broadcasting for further analysis,
  %       (4) \texttt{CommonSubexpElim}, which lifts common subexpressions.
  % \end{itemize}

  % Figure~\ref{fig:opt-eval} shows mean inference speedup relative to
  %   \texttt{-O0} as Relay applies optimizations more aggressively.
  % Average performance improves by up to 2$\times$ when all optimizations are applied.
  % Most networks benefit greatly from operator fusion.
  % Nature-DQN~\citep{dqn} has simple operators, which don't benefit from optimizations
  %   such as layout transform, explaining why its performance doesn't improve beyond \texttt{-O1}.
  % ResNet-18~\citep{resnet} and VGG-16~\citep{vgg} are two dense convolutional neural
  %   networks which benefit from \texttt{-03} optimizations.
  % These networks contain dense \texttt{conv2d} operators that benefit
  %   from the \texttt{AlterOpLayout} pass.
  % Overall, these results show that Relay lets us compose optimizations
  %   in a way that is beneficial to diverse workloads.

  \subsection{Relay Provides Competitive Performance}
  \label{sec:perf-gpu}

  \begin{figure}[h]
    % \includegraphics[width=0.6
    % \textwidth]{fig/eval/vision_1080Ti_relay.pdf}
    % \captionof{figure}{
    %   Inference slowdown of popular frameworks relative to Relay on vision
    %     benchmarks running on NVIDIA GTX 1080 Ti GPUs.
    %   Relay provides performance competitive to the state of the art.
    %   We ran 1000 trials for each model and used the AoT compiler.
    % }
    % \label{fig:vision-eval}
  \end{figure}

  \begin{figure}[h]
    % \includegraphics[width=0.6
    % \textwidth]{fig/eval/nlp_TitanV_relay.pdf}
    % \captionof{figure}{
    %   Inference slowdown relative to Relay on NLP benchmarks running on NVIDIA
    %     Titan-V GPUs.
    %   NLP workloads feature control flow,
    %     which makes them more challenging to optimize.
    %   Relay provides performance competitive to state of the art (up to
    %     2.4$\times$ speedup over MxNet on GRU).
    %   We ran 1000 trials for each model, except for CharRNN, on which we used 100 trials.
    % }
    % \label{fig:nlp-eval}
  \end{figure}

  An age-old story in compilers literature is that increasing expressivity
    impacts the global performance of the system.
  We set out to build zero-cost abstractions for Relay,
    governed by Stroustrup's principle, ``What you don't use, you don't pay
    for'' \citep{bjarne}.
  We demonstrate that we can achieve competitive performance on both CPUs and
    GPUs on a wide set of CNNs that are well supported by existing frameworks.
  We evaluated inference time for two classes of workloads: computer vision and natural language processing.
  We compared Relay (using our AoT compiler) to \nnvm,
    TensorFlow, TensorFlow-XLA (Accelerated Linear Algebra), PyTorch, and MxNet.
  We ran the vision and NLP workloads on GTX 1080 Ti and Titan-V GPUs, respectively.

  \paragraph{Vision Evaluation}
  % Figure~\ref{fig:vision-eval} compares Relay against state of the art frameworks
  %   running vision workloads on a GTX 1080 Ti GPU.
  We ran each model with
    batch size 1, a common setting in inference tasks.
  Relay achieves performance on par with \nnvm,
    an existing deep learning graph compiler in use at Amazon.
  Relay outperforms TensorFlow, TensorFlow-XLA, MxNet and
    PyTorch on every benchmark.
  Relay's ability to do aggressive optimizations like operator
    fusion on long chains of operations, generating hardware
    specific implementations, enables it to outperform
    existing frameworks that don't perform inter-operator optimizations.

  \paragraph{NLP Evaluation}
  % Figure~\ref{fig:vision-eval} compares Relay against state-of-the-art NLP models on a Titan-V GPU.
  % Implementations of the NLP models were not available in all frameworks;
    we used MxNet baselines for RNN, GRU, and LSTM and PyTorch for Char-RNN and TreeLSTM.
  % We ran the models for 1000 iterations per input, except char-RNN, which we ran for 100 ???.
  % To run the RNN, GRU, and LSTM benchmarks in MxNet, and Char-RNN, and TreeLSTM
  %   in PyTorch.
  Relay performs better than MxNet on recursive models
    due to the fact they are implemented in Python using
    MxNet's looping constructs.
  PyTorch instead uses handwritten and heavily optimized
    C implementations of the recursive network cells.
  Due to this we perform slightly \emph{worse} than PyTorch.
  It is interesting to note that our pure Relay
    implementation performs competitively against
    the hand-optimized version.

  \subsection{Relay Handles Challenging Backends}
  \label{sec:low-power}

  \begin{figure}[h]
    % \includegraphics[width=\textwidth]{fig/eval/vision_arm.pdf}
    \caption{
      Inference time (ms) of vision DNNs on low-power platforms using
        different data types.
      Relay allows us to reduce inference time on power-constrained devices by
        easily substituting \texttt{float32} multiplications with \texttt{int8}
        multiplications and \texttt{int16} or \texttt{int32} accumulations (denoted
        at \texttt{int8}/\texttt{int16} and \texttt{int8}/\texttt{int32} respectively).
      We used 1000 trials for each model.
    }
    % \label{fig:arm-eval}
  \end{figure}

  \begin{figure}[h]
    % \includegraphics[width=0.6\textwidth]{fig/eval/vision_fpga.pdf}
    \caption{
      Inference time (ms) of vision DNNs on Ultra-96 FPGA-enabled SoC.
      We compare vision workloads that Relay compiles onto the embedded Cortex
        A53 CPU vs. a DNN accelerator implemented on the integrated FPGA fabric.
      Targeting DNN accelerators can unlock up to 11x speedups, but requires a
        multitude of graph-level transformations.
      We used 10 trials for each model.
    }
    % \label{fig:fpga-eval}
  \end{figure}

  Relay can handle challenging scenarios: consider edge inference where energy is a first order
    constraint due to thermal limitations or limited battery life.
  One option is to apply more aggressive quantization: instead of performing expensive
    arithmetic in the floating point domain, simpler and narrower fixed point data is used.
  Another option is hardware acceleration: instead of evaluating
    compute-intensive operations on the CPU, we can offload to a specialized accelerator.

  \paragraph{Quantized Inference on ARM CPUs and GPUs}
  We evaluate the effects of quantized inference applied by Relay on vision workloads running
  on the Raspberry-Pi3 and Firefly RK3399 ARM-based platforms.
  % Figure~\ref{fig:arm-eval} shows the effects of different levels
  %   of quantization applied to low-power devices~\citep{roesch2019relay}.
  % The numbers show that as we opt for a more aggressive quantization scheme such as \texttt{int8/16} (i.e. 8-bit multiplication and 16-bit accumulation), we achieve much improved performance.

  \paragraph{Targeting Deep Learning Accelerators on FPGAs}
  We evaluated inference time on five models including MobileNet-G \citep{mobilenet}, a grouped variant of the MobileNet architecture; ResNet-18, ResNet-34, and ResNet-50\citep{resnet}; and Deep Convolutional Generative Adversarial Networks \citep{dcgan}, a generative DNN used in unsupervised learning.
  Overall, Relay helps us efficiently offload deep learning operators onto specialized accelerators like VTA.
  Our results in Figure~\ref{fig:fpga-eval} show that we can achieve between 2.5 to 11.7$\times$ reduction in single-batch inference latency by offloading critical operators to the FPGA accelerator.
  These experiments demonstrate Relay's ability to target current and future deep learning architectures:
  \begin{enumerate}
    \item \textit{Heterogeneous FPGA/CPU offloading}: Relay lets us define the rules for offloading specific operators to the FPGA-based accelerator.
    \item \textit{Push-button quantization}: Relay can take a \texttt{fp32} model and convert its parameters to \texttt{int8} in order to enable inference on specialized accelerators.
    \item \textit{Accelerator-friendly data packing:} Relay reorganizes data so it can be effortlessly consumed by a specialized TPU-like accelerator~\citep{tpuv1}.
  \end{enumerate}

The scenario presented in the introduction demonstrates the three-pronged \textbf{extensibility challenge}
  for DL IRs:
% \begin{enumerate} % [label=\arabic*.]
%   \item \textit{Expressivity}: It should be straightforward to write models involving complex data structures (e.g., trees, graphs, and lists) and control flow.
%   \item \textit{Composability}: It should be straightforward to add and compose new optimizations
%     with existing ones (e.g., quantization, operator fusion, and automatic differentiation).
%   \item \textit{Portability}: It should be straightforward to add new hardware backends
%     (e.g., TPU, Inferentia, and FPGAs)~\citep{tpuv1, inferentia}.
% \end{enumerate}

