\begin{figure}
  \begin{minted}{ocaml}
             const(v, sh, dt)
  \end{minted}
  $\Downarrow$
  \begin{minted}{ocaml}
    (const(v, sh, dt), ref(const(0, sh, dt)))
  \end{minted}
  \vspace{0.25cm}
  \begin{minted}{ocaml}
    op : fn<>(a_1 : T1, ..., a_n : Tn) -> O
  \end{minted}
  $\Downarrow$
  \begin{minted}[mathescape=true]{ocaml}
def @op_grad<>(a_1 : type_transform(T1), ...,
    a_n : type_transform(Tn))
    -> type_transform(O) {
  let %r = rev_op<>(a_1, ..., a_n);
  let %ref = ref(zeros_like<>(r.0));
  let %bpv = !bp;
  let %nbp = def @bpf<>() -> () {
    let %grad = (r.1)(!ref);
    a_1.1 := add<>(!(a_1.1), grad.0);
    ...
    a_n.1 := add<>(!(a_n.1), grad.(n-1));
    bpv()
  }
  bp := nbp;
  (r.0, ref)
}
  \end{minted}
  \caption{
    The above depicts the AD expression
      transformation macro.
    The macro takes as an
      argument an expression AST and a backpropagator
      \texttt{bp}
      (a reference to a function that takes no
      arguments and returns unit,
      which serves to update the reverse value references)
      and recurses down the expression's AST,
      making the above substitutions wherever
      it encounters a tensor constant or an operator
      \texttt{op}.
    All other AST constructs are unchanged by the macro.
    For simplicity, type arguments and relations
      are omitted in this figure;
      at present, \relay does not
      support type arguments in gradients.
    Note that the macro \texttt{type\_transform}$(T)$
      similarly recurses down the AST of the type $T$
      and replaces all tensor types with a
      pair of a tensor type
      and a reference to a tensor type
      (the first member is the forward value;
      the second, a reference to the reverse value)
    For a given operator \texttt{op},
      \texttt{rev\_op} refers to the registered reverse function
      (the registration is described in Sec.~\ref{sec:growing}).
    Additionally, for brevity,
      a semicolon without a \texttt{let}
      is used above as a sequence operator,
      namely a let node in which the bound
      variable is disregarded.}
  \label{fig:ad-expr-transform}
\end{figure}

\begin{figure}
  \begin{minted}{ocaml}
def @grad_f<>(a_1 : T1, ..., a_n : Tn)
    -> (O, (T1, ..., Tn)) {
  let %bp = ref(def @bpf<>() -> () { () });
  let %p_1 = (a_1, ref(zeros_like<>(a_1)));
  ...
  let %p_n = (a_n, ref(zeros_like<>(a_n)));
  let %c = (autodiff(bp, f))<>(p_1, ..., p_n);
  c.1 := ones_like<>(c.0);
  (!bp)();
  (c.0, (!(p_1.1), ..., !(p_n.1)))
}
  \end{minted}
  \caption{The gradient \relay{}'s AD produces
    for a function \texttt{f} of type
    \texttt{fn}$\langle\rangle(T_1, \ldots T_n) \rightarrow O$.
  The gradient function returns a tuple of
    the forward value of the function
    (simply evaluating the function at the arguments)
    and a tuple containing the
    partial derivative with respect to each argument.
  \texttt{autodiff} refers to the macro described in
    Fig. \ref{fig:ad-expr-transform}.
  As with Fig. \ref{fig:ad-expr-transform},
    the present implementation of
    \relay{}'s AD does not handle type relations or type arguments,
    so they are omitted in this figure.
  Also as in Fig. \ref{fig:ad-expr-transform},
    semicolons without \texttt{let} bindings
    are used as sequence operators for convenience.}
  \label{fig:ad-gradient}
\end{figure}
