\chapter{ML Frameworks}
\label{ch:frameworks}

Using Relay as a building block for ML frameworks.


\subsection{Beacon}

A related challenge is increasing the ability for users to describe models which take
  full advantage of \relay's more expressive semantics.
As discussed in \ref{ch:related} existing frontends have been designed to capture the
  subset of programs which be optimized well, and designs have ignored aspects of the
  programs that are not well optimized.
In order to demonstrate what we can do with greater access to the program we constructed
  Beacon a PyTorch-like inspired framework which uses TVM for all execution.

Beacon uses a novel design related to ideas from previous work detailed in \ref{ch:related}
  in order to convert Python directly into IR which can be compiled by \relay.
Beacon simply defines a tensor which stores either a value or a partial program
  which can be used to execute the program.
Beacon's core abstraction can be summarized in a small fragment of code, seen below.

\begin{figure}
\begin{minted}{python}
  class Tensor:
    def __init__(self, expr, value=None):
        self.expr = expr
        self._value = value

    def value(self, evaluator="debug"):
        mod = get_module()
        ex = relay.create_executor(kind=evaluator, mod=mod)
        inputs = relay.ir_pass.free_vars(self.expr)
        binds = dict((var, _lookup_var(var)._value) for var in inputs)
        return ex.evaluate(self.expr, binds).data
\end{minted}
\end{figure}

Because Beacon serves as a thin abstraction on top of Relay it
  is easy to try out new approaches for generating IR, and build
  higher level abstractions, such as iterators which construct
  loopy IR, or using program rewriting techniques like those
  used in AutoGraph~\citep{AutoGraph}.
Users can also use pure Python to construct arbitrary Relay
  programs which can then be ahead of time optimized and
  deployed.
Beacon serves as a platform for experimenting with how
  to use the IR, and inform optimizations and accelerator optimizations.
Beacon is just a starting point, we will discuss how to enable this for frameworks
  such as PyTorch and MxNet in Section \ref{sec:future}.
