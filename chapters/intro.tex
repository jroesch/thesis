\chapter{Introduction}
\label{ch:intro}

Traditionally machine learning algorithms were carefully engineered by humans
  in order to leverage limited data to obtain sufficient task performance.
During the Big Data era, these hand engineered algorithms were displaced by
  simpler algorithms applied to the abundantly available data [1].
In order to scale these algorithms a number of systems were
  designed most notably streaming and graph processing systems such as MapReduce, and Spark.
In the early 2010s the emergence of deep learning changed the paradigm again
  replacing simple regular computation over streams of data with large batched-models
  composed of compute-hungry tensor computations.
This trend towards models which are applied to more complex data (i.e., full text documents, images, videos, audio)
  process a greater quantity of data (millions of data points), and are much more expensive to evalute
  on a single datum (i.e., terra-flops of compute per sample).

This new generation of machine learning applications demanded a corresponding era of
  innovation in systems.
The first innovation in deep learning systems were so called
  ‘deep learning frameworks’ such as TensorFlow and PyTorch.
These monolithic systems can execute a subset of deep learning models
  efficiently on a subset of the available hardware devices.
These rigid systems provide good baseline performance, but often fail
  when straying from well-used models.

Scaling frameworks to span the scope of all models and targets is a human endeavor
  requiring deep expertise often not found in conjunction with the skills of a
  machine learning engineer.
A single machine learning engineer cannot write code that scales across the growing
  sea of models, operators, optimizations, and devices while maintaining state-of-the-art
  performance.
From the early days of deep learning framework development researchers realized the
  potential for apply compilation to this problem, systems such as Theano were applying
  this as earlier as 2007.

Deep learning compilers, systems for performing compilation to accelerate the execution
  of deep learning models, have made similar tradeoffs as frameworks accelerating
  a subset of models on a subset of devices.
Designing a deep learning compiler even for this constrained subset is challenging.
For example,
  a popular state-of-the-art DL compiler,
  Google's XLA, can famously slow down a programs on
  CPU instead of speeding them up (cite needed).

Compilers for deep learning are a rapidly evolving area explored by a
  variety of solutions both industrial and academic.
Part of my thesis work in graduate school
  was meaningful contributions to the design of and implementation of deep learning compilers.
Much of this work now exists in TVM a production grade deep learning compiler.

In particular I have spent the last several years focused on the
  representation,
  optimization,
  differentiation,
  and execution of dynamic neural networks.
In this thesis I propose that we can generalize overspecialized optimizations
  and insights applied to static dataflow graphs, the predominant programming model
  of deep learning to fully dynamic neural networks.
These generalizations are powered by a simple insight:
  dynamic neural networks are just programs which manipulate tensors.

The challenge is how to build a representation that captures this generality
  in a principled way allowing us to recover the potential performance loss by generalizing
  the programming model.

When I began working on deep learning compilers, input models were high-level
  declarative programs composed of linear algebra and statistics primitives.
These straight forward representations often appeared in the form of
  a directed acyclic dataflow graph.

We realize a unified interface in the form of Relay a set of systems and APIs designed
  to compose together techniques across the stack to achieve state of the art performance.

Instead of building a single IR and compiler to rule them all we carefully split
  the program optimization problem into a series of phases each focused on a specific
  optimization task.
By doing this we are able to balance the tension between expressivity, composition
  and performance portability.
Relay is a part of the TVM open source project, and catalystized a major redesign of
  TVM, in order to embrace and extend the ideas of this thesis.
Relay is deployed in production at multiple large companies including
  OctoML, Amazon, Facebook, and Microsoft.
One notable impact is its use in
  optimizing Alexa’s wake word model, executed each time a user interacts with
  Alexa.

\section{State of the Art}

Popular DL compiler intermediate representations (IRs) offer different tradeoffs
  between expressivity, composability, and portability~\citep{
    tensorflow, pytorch_ad, chainer_learningsys2015, tangent, theano, glow}.
Early frameworks adopted IRs
  specialized for then-state-of-the-art models and/or
  emerging hardware accelerators.
As a result, non-trivial extensions require
  patching or even forking frameworks~\citep{
    tf_fold, tf_lite, tangent, tf_eager, xla, glow, torchscript}.
Such \textit{ad hoc} extensions can improve expressivity
  while maintaining backwards compatibility with existing execution mechanisms.
However, they are difficult to design, reason about, and implement,
  often resulting in modifications that are mutually incompatible.

Let us consider a hypothetical scenario that exemplifies
  IR design tensions in DL compilers.
Suppose a machine learning engineer wants to write
  an Android app that uses sentiment analysis to
  determine the moods of its users.
To maintain privacy, the app must run completely on-device,
  i.e., no work can be offloaded to the cloud.
The engineer decides to use a variant of TreeLSTM,
  a deep learning model that uses a tree structure~\citep{tree_lstm}.
Unfortunately, current frameworks' IRs cannot directly encode trees,
  so she must use a framework extension
  like TensorFlow Fold~\citep{tensorflowfold}.

Suppose that after adapting the model to run on her phone,
  the out-of-the-box performance of her
  model on her particular platform is not satisfactory, requiring her to optimize it.
She chooses to employ \textit{quantization}, an optimization that
  potentially trades accuracy for performance by replacing
  floating-point datatypes with low-precision ones.
Although researchers have developed a variety of quantization
  strategies, each of which makes use of different bit-widths, rounding
  modes, and datatypes, our engineer must use a strategy supported
  by existing frameworks~\citep{gustafson2015end, tf_lite_ops_compat, glow_quant}.
Unfortunately, frameworks only provide support for a small number
  of strategies, and supporting new quantization strategies is non-trivial.
Each combination of operator, datatype, bit-width, and
  platform requires unique operator implementations.
Optimizations like operator fusion exacerbate this combinatorial explosion,
  further increasing
  the number of unique implementations required.
Furthermore, if a framework doesn't have specific support for
  the target phone model she cannot take advantage of specialized deep learning
  instructions or coprocessors~\citep{apple_neural_engine}.

The state of the art when I began my thesis work were
  deep learning frameworks mostly had a single "lego" style design.
Each technology stack had deep learning frameworks at the top,
  layered from the bottom from high-performance,
  device specific kernel libraries provided by hardware manufacturers.

As deep learning frameworks began to evolve deep learning compilers
  the compiler design was constrained the heavily layered abstractions
  and the programming model.
An example of a constraining abstraction is the reliance on individual
  kernels which can be invoked at the bottom most layer.
By allowing the lowest level of abstraction to only receive a single
  kernel invocation you limit the ability to optimize as the final
  program must map directly to existing kernels which when possible
  may be sub-optimal.
Practically this has introduced a strict separation between so-called
  ‘graph-compilers’ and so called ‘kernel libraries’.
These graph compilers implement a simple DAG of operations
  which they then map down to kernels.
The graph compilers are a passable solution, assuming the user can describe their program in a restrictive and awkward language,
  and only makes use of existing kernels.
If the user model makes use of the well-optimized subset of kernels and well-optimized devices it can assume passable performance,
  but with many models this is not that case, with very little recourse for program authors.

In the kernel domain there has been a large amount of existing work on
  ensuring individual kernels, or pipelines of them are efficient,
  but these DLSs are not integrated into a larger programming abstraction.
There existing systems like TensorRT which solve this problem for a specific device,
with a series of caveats.

Compiler design using the first strategy are limited by each layer only having an opaque view of the layers above and below,
each layer uses a single semantics with well-defined mappings between layers but limited communication between,
 or in the case of a single monolithic IR, which chooses a functional semantics the inability to use semantics which provide
 hardware specificity or hardware aware semantics.

Imagine a not-so-hypothetical end-user’s desire to support a new Transformer based model on a previously unsupported device such as a microcontroller.
In order to obtain best in class performance this task will be split across an engineering team.
For example one user with the correct expertise may contribute code for importing this model into, another with kernel expertise separately implements
a kernel such as batched matrix multiplication and then optimizes it, and another user contributes the code generation, and another the runtime support.

To accomplish this task in today’s ML frameworks a user must first design and train a model which accomplishes the task, k
now how to use framework tools such as XLA, or TorchScript, be able to modify their complex source bases, and write a complex code generator
or LLVM backend and finally modify the runtime to execute on the device, if even possible.

Many engineers in this domain believe that human effort is required to obtain competitive performance that is demonstrated by today’s bleeding edge development,
my thesis demonstrated this is not the case.

\section{Dynamic Neural Networks}

In particular all previous solutions failed to provide the same optimizations
  level of performance in the presence of dynamic features.

In this thesis we explore new designs for representing,
  optimizing, differentiating, and executing dynamic neural networks.
Existing systems for compiling neural networks do not provide robust
   solutions that achieve state of the art performance on dynamic models.
Most solutions for dynamic models either
  a) use non-shape specialized kernels b) trace based JIT compilation.

Relay addresses the challenges laid out in the previous section by using
  a whole program representation which composes multiple distinct IR dialects,
  notably an ML-style high-level dialect,
  and the use of TVM’s compute language as a low-level dialect,
  in conjunction with carefully designed runtime and system APIs.

The hypothetical story of the user needing to extend the entire stack is derived
  from a true sequence of events that occurred in the development of Relay.
A process like this requires diverse expertise that is often non-compositional. It is nearly impossible to find a single individual who has the ability
  to do all these tasks at a state-of-the-art level.
One user introduced framework support, another added a kernel, another optimized it, and another added compiler and runtime support for a new target.

We were able to do this in a way where each user was able to make these changes independently and then compose them.

Relay is able to provide this by introducing a set of key interfaces, a high-level extensible IR which can be used to encode various IR dialects,
  for example we can leverage Relay’s type system to provide the illusion of functional operators until a later dialect in which we change
  to a destination passing stye.

Although this work began in 2017, the introduction of MLIR project (2018), a framework for composing layered dialects with different semantics,
is one validation point of Relay’s design.

If we attempt to use existing end-to-end tools such as Lift, they provide a single functional unified-semantics.

By embracing a split semantics, with the ability to introduce new dialects we provide different views of the program at different points in order to solve each optimization task independently.

This work is focused on helping a user wishing to utilize machine learning specify and optimize
  the neural networks they wish too with minimal fuss and maximal performance.

\section{Thesis Organization}

My thesis is organized around four pillars, representation, differentiation, optimization, execution,
We first explore related work and background material
  in \ref{ch:related}, we then discuss the design of the Relay intermediate representation in \ref{ch:relay}.
We then discuss how to perform automatic differentiation and training support in \ref{ch:ad} and
  \ref{ch:frameworks}.
We then discuss how to optimize the compiler in \ref{ch:optimizations} and \ref{ch:opt_for_non_experts}.
We then discuss how to lower, execute, and compile these programs in
\ref{ch:compiler}, \ref{ch:dynamic}.
Finally we look to \ref{ch:future} to discuss future work and extensions to this work.


% Bibliography

% [1] "We don’t have better algorithms. We just have more data.”

% [2] data and model size the amount of compute is exponentially growing, roughly doubling every 3.4 months (see https://twitter.com/OpenAI/status/1192481690741903360).

% [3] https://arxiv.org/pdf/1911.05289.pdf

% THESES TO LOOK AT:
% https://people.csail.mit.edu/jrk/jrkthesis.pdf
% https://homes.cs.washington.edu/~djg/theses/sampson_thesis.pdf
% https://llvm.org/pubs/2005-05-04-LattnerPHDThesis.pdf
